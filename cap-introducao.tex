%% ------------------------------------------------------------------------- %%
\chapter{Introdução}
\label{cap:introducao}


Neste trabalho abordarei sobre um importante tópico da Teoria de Grafos: o ancestral 
em comum mais próximo - ou simplesmente LCA (derivado do inglês \emph{Lowest Common 
Ancestor}).

A motivação para estudar este assunto veio majoritariamente do meu engajamento na 
Maratona de Programação durante toda a graduação. Nestas competições os participantes 
são expostos a desafios lógicos que requerem a escrita de códigos que, dada uma 
entrada para certo problema, são capazes de retornar sua resposta.

Entender o problema de encontrar o LCA entre dois vértices de um grafo não é uma 
tarefa complicada. Ademais, escrever um algoritmo simples para tal também não possui 
um grau de complexidade elevado. Entretanto, nessas competições sempre procuramos 
resolver os problemas de forma eficiente, visto que o tamanho da entrada pode ser 
grande o suficiente para exigir um algoritmo rápido (e, geralmente, mais complexo) 
que termine sua execução dentro do tempo limite proposto.

Diversos algoritmos são conhecidos para encontrar o LCA - uns mais eficientes do 
que outros. Ao longo deste trabalho me aprofundarei em algumas soluções, partindo 
de algoritmos mais simples porém ineficientes até uma solução eficaz e mais complexa.

O trabalho será abordado com viés educacional, visando o aprendizado do leitor. 
Para isso, ao apresentar algoritmos sempre procuro fornecer códigos e listar 
alguns problemas interessantes de juízes \emph{online} que exploram o tema estudado.


\section{Organização do texto}

O capítulo 2 aborda conceitos importantes para o entendimento do trabalho.

O capítulo 3 aborda etc etc etc.


% TEMPLATE:

\iffalse
Uma monografia deve ter um capítulo inicial que é a Introdução e um
capítulo final que é a Conclusão. Entre esses dois capítulos poderá
ter uma sequência de capítulos que descrevem o trabalho em detalhes.
Após o capítulo de conclusão, poderá ter apêndices e ao final deverá
ter as referências bibliográficas.


Para a escrita de textos em Ciência da Computação, o livro de Justin Zobel, 
\emph{Writing for Computer Science} \citep{zobel:04} é uma leitura obrigatória. 
O livro \emph{Metodologia de Pesquisa para Ciência da Computação} de 
\citet{waz:09} também merece uma boa lida.

O uso desnecessário de termos em lingua estrangeira deve ser evitado. No entanto,
quando isso for necessário, os termos devem aparecer \emph{em itálico}.

\begin{small}
\begin{verbatim}
Modos de citação:
indesejável: [AF83] introduziu o algoritmo ótimo.
indesejável: (Andrew e Foster, 1983) introduziram o algoritmo ótimo.
certo : Andrew e Foster introduziram o algoritmo ótimo [AF83].
certo : Andrew e Foster introduziram o algoritmo ótimo (Andrew e Foster, 1983).
certo : Andrew e Foster (1983) introduziram o algoritmo ótimo.
\end{verbatim}
\end{small}

Uma prática recomendável na escrita de textos é descrever as legendas das
figuras e tabelas em forma auto-contida: as legendas devem ser razoavelmente
completas, de modo que o leitor possa entender a figura sem ler o texto onde a
figura ou tabela é citada.  

Apresentar os resultados de forma simples, clara e completa é uma tarefa que
requer inspiração. Nesse sentido, o livro de \citet{tufte01:visualDisplay},
\emph{The Visual Display of Quantitative Information}, serve de ajuda na
criação de figuras que permitam entender e interpretar dados/resultados de forma
eficiente.
\fi

