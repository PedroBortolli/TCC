%% ------------------------------------------------------------------------- %%
\chapter{Conceitos importantes}
\label{cap:conceitos-importantes}

Neste capítulo serão apresentados brevemente alguns conceitos de suma importância 
para a compreensão do trabalho. Entretanto, já é suposto que o leitor tenha um 
conhecimento básico sobre os temas listados a seguir. O leitor que não estiver familiarizado com tais assuntos deve buscar um texto básico, como o livro \cite{cormen}.

\section{Grafo}

Conjunto de vértices (também chamados de nós) conectados por arestas.

\section{Caminho}

Sequência de arestas que ligam vértices de um grafo. Em um caminho é possível 
partir de um vértice $a$ e chegar em um vértice $b$, usando em cada passo uma aresta cujas pontas são os vértices vizinhos pertencentes ao caminho.

\section{Árvore}

Caso especial de um grafo, onde para quaisquer 2 vértices existe apenas um único caminho que os liga no grafo. Dizemos que uma árvore é enraizada se escolhemos um vértice para ser a raiz e todas as arestas partirem dele para seus descendentes (neste sentido).

\section{Profundidade} 

A profundidade de um vértice em uma árvore enraizada é dada pelo tamanho de seu caminho
até a raiz - isto é, quantidade de arestas pertencente à este caminho . A profundidade da raiz é sempre 1.

\section{Arestas direcionadas}

Um grafo pode ter as suas arestas direcionadas ou não. Neste texto, todas as árvores apresentadas são não direcionadas. Entretanto, pode-se sempre fixar um vértice para ser a raiz e criar um direcionamento a partir dele. Assim, todos os algoritmos apresentados neste trabalho funcionam para grafos direcionados e não direcionados.

\section{Complexidade de um algoritmo}

Existem algumas definições diferentes para medir a complexidade de um 
algoritmo, acompanhadas de notações distintas. Por exemplo, pode-se avaliar um 
algoritmo por seu melhor caso, médio ou pior. O que mais nos interessa neste 
trabalho é a análise de pior caso, e para isso usa-se a notação \emph{"Big O"}.

\section{Dicionário}

Conjunto de pares de chaves e valores. Para cada chave, existe apenas um valor associado a ele. Podem ser de qualquer tipo, embora neste trabalho sempre utilizaremos um mapeamento de inteiros para inteiros.