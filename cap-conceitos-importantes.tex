%% ------------------------------------------------------------------------- %%
\chapter{Conceitos importantes}
\label{cap:conceitos-importantes}

Neste capítulo serão apresentados brevemente alguns conceitos de suma importância 
para a compreensão do trabalho. Entretanto, já é suposto que o leitor tenha um 
conhecimento básico sobre os temas listados a seguir.

\section{Grafo}

Conjunto de vértices (também chamados de nós) conectados por arestas. \cite{cormen}

\section{Caminho}

Conjunto de arestas que ligam vértices de um grafo. Em um caminho é possível 
partir de um vértice $a$ e chegar em um vértice $b$, sem revisitar nenhum 
vértice pertencente ao caminho ou caminhar sobre a mesma aresta mais de uma vez.

\section{Árvore}

Caso especial de um grafo, onde quaisquer 2 vértices estão conectados por apenas 
um único caminho.

\section{Profundidade} 

A profundidade de um vértice em uma árvore enraizada é dada pelo tamanho de seu caminho
até a raiz - isto é, quantidade de vértices pertencente à este caminho . A profundidade da raiz é sempre 1.

\section{Arestas direcionadas}

Um grafo pode ter as suas arestas direcionadas ou não. Neste texto, todas as árvores apresentadas são não direcionadas. Entretanto, pode-se sempre fixar um vértice para ser a raiz e criar um direcionamento a partir dele. Assim, todos os algoritmos apresentados neste trabalho funcionam para grafos direcionados e não direcionados.

\section{Complexidade de um algoritmo}

Existem algumas definições diferentes para medir a complexidade de um 
algoritmo, acompanhadas de notações distintas. Por exemplo, pode-se avaliar um 
algoritmo por seu melhor caso, médio ou pior. O que mais nos interessa neste 
trabalho é a análise de pior caso, e para isso usa-se a notação \emph{"Big O"}.

\subsection{Notação \emph{Big O}}
Descreve o tempo ou espaço que um algoritmo consome para ser executado no 
pior caso (ou seja, dentre todas as entradas possíveis para um programa, aquela 
que resulta em seu pior desempenho). Nota-se por um $O$ seguido de um número real. 

Por exemplo, usando tal notação, a complexidade de tempo de um algoritmo que 
executa um laço \emph{for} de $n$ iterações para somar números, por exemplo, é 
$O(n)$, já que no pior caso (em especial, em todos) o algoritmo executa $n$ 
operações.

\subsection{Uma definição mais formal}
Vamos usar um exemplo para medir a complexidade de tempo de um algoritmo. 
Entretanto, como mencionado anteriormente, a ideia é a mesma para medir sua 
complexidade de espaço.

Seja $Q(n)$ uma função que expressa a quantidade de operações que um algoritmo 
realiza. Podemos dizer que $Q(n)$ é $O(f(n))$ se existe uma constante $c > 0$ 
tal que $Q(n) \leq c f(n)$ para todo $c$ suficientemente grande.

\hspace{1cm}

Considere o seguinte algoritmo:

\begin{algorithm}
\caption{Algoritmo simples com dois laços for encadeados}\label{euclid}
\begin{algorithmic}[1]

\State $soma$ = 0
\For{$i$ = 1 to 5}
    \For{$j$ = 1 to $n$}
        \State $soma$ += $j$
        \If{$soma$ > $somaMaxima$}
        \State \Return
        \EndIf
    \EndFor
\EndFor
\end{algorithmic}
\end{algorithm}

O número total de operações executadas por este algoritmo depende do valor 
da variável $somaMaxima$. Quando $somaMaxima$ = 0, apenas uma operação será 
realizada. Entretanto, como dito anteriormente vamos analisar um algoritmo 
sempre no seu pior caso, que neste exemplo ocorre quando o valor de $somaMaxima$
é grande o suficiente para que $soma$ nunca o atinja.

Portanto, podemos dizer que este algoritmo executa $10n$ operações. Assim, o 
algoritmo é $O(n)$, pois já que $f(n) = n$ e ao escolher uma constante $c = 10$ 
temos que $10n \leq 10n$.
