%% ------------------------------------------------------------------------- %%
\chapter{Decomposição em Raiz}
\label{cap:decomp-raiz}

\section{Introdução}

Na seção anterior desenvolvemos um algoritmo simples que percorre, no pior caso, todos os vértices de uma árvore. Neste capítulo apresentaremos uma otimização ao algoritmo anterior para que sua complexidade de tempo seja $O(sqrt(n))$, onde \textit{sqrt} remete-se ao termo inglês \textit{square root} - ou seja, raiz quadrada de $n$.

\section{Otimização}

A ideia do algoritmo que vamos desenvolver é a mesma descrita no Algoritmo 3. Isto é, para dois vértices $a$ e $b$ ainda vamos percorrer seus caminhos até a raiz para determinar o seu \LCA.

A diferença é que agora não vamos potencialmente visitar todos os vértices do caminho até encontrar o \LCA. Agora, vamos dividir os vértices da árvore em "baldes", para que inicialmente um pulo entre dois membros do caminho seja de um vértice de um balde para um vértice de outro balde, ao invés de ir para o seu pai direto. Assim, um pulo não tem mais tamanho 1, mas sim $b$, onde $b$ é a altura de um balde.

\subsection{Detalhando o algoritmo}

Assumindo que as profundidades dos vértices já foram calculadas, vamos criar alguns baldes de forma que:

\begin{itemize}
    \item O balde número 1 tenha todos os vértices de profundidade [$1$...$p$];
    \item O balde número 2 tenha todos os vértices de profundidade [$p+1$...$2p$];
    \item O balde número 3 tenha todos os vértices de profundidade [$2p+1$...$3p$];
    \item etc.
\end{itemize}

Além disso, precisamos guardar, para cada vértice de um balde, quem é o seu ancestral mais profundo do balde anterior. Para melhor ilustrar isso, apresentamos a figura x:

<figura>

Então, para encontrar o \LCA\ entre dois vértices $a$ e $b$ basta verificar em qual balde ele está (isto é, percorrer os caminhos de $a$ e $b$ para a raiz até que os ancestrais do balde anterior sejam iguais), e daí trivialmente encontrá-lo.

\section{Código}

Utilizaremos os algoritmos 2 e 3 demonstrados anteriormente. Veremos que precisamos de poucas adições neles para que fiquem otimizados.

\hspace{1cm}

Para a função de cálculo de profundidade, também guardaremos quem é o seu ancestral mais profundo do balde anterior. Além disso, toda vez que nível for múltiplo da altura do balde permitido "criaremos" um novo balde.

\begin{algorithm}[H]
\caption{Modificação do algoritmo 2}
\begin{algorithmic}[1]
\Function{\textsc{CalculaProfundidade}}{vertice,\ nivel,\ ancestralAnterior}
    \State $profundidade[vertice] \rec nivel$
    \State $baldeAnterior[vertice] \rec ancestralAnterior$
    \If{$nivel \ \% \ alturaBalde = 0$}
        \State $anterior \rec vertice$
    \Else
        \State $anterior \rec ancestralAnterior$
    \EndIf
    \For{cada $filho$ em $filhos[vertice]$}
        \State $CalculaProfundidade(filho,\ nivel+1,\ anterior)$
    \EndFor
\EndFunction
\end{algorithmic}
\end{algorithm}

Para a função que determina o \LCA, primeiro faremos com que $a$ e $b$ estejam no mesmo balde cujo ancestral anterior seja o mesmo para ambos. Após isso, basta chamar a função de \LCA\ simples, demonstrada no algoritmo 3.

\begin{algorithm}[H]
\caption{\LCA\ utilizando o conceito de baldes}
\begin{algorithmic}[1]
\Function{\textsc{LCA}}{a,\ b}
    \While { $baldeAnterior[a]$ != $baldeAnterior[b]$}
        \If{$profundidade[a] < profundidade[b]$}
            \State $troca(a,\ b)$
        \State $a \rec baldeAnterior[a]$
        \EndIf
    \EndWhile
    \\\hspace{5mm} \Return $LCA\_Simples(a,\ b)$
\EndFunction
\end{algorithmic}
\end{algorithm}

\subsection{Corretude}

A ideia deste algoritmo é similar à apresentada no algoritmo 3: após a execução do laço da linha 2 teremos dois vértices no mesmo balde cujo ancestrais do balde anterior são os mesmos. Afinal, sabemos que percorrer o caminho de um vértice pulando de balde em balde resultará no mesmo destino do algoritmo 3, que caminha vértice após vértice: a raiz. Isso se deve ao fato de que no algoritmo 4 apresentado nessa seção, o objeto $baldeAnterior$ apenas serve como um encurtador de caminhos - mais precisamente deixando cada pulo em um caminho de  tamanho $alturaBalde$ (isto é, a partir de um vértice de profundidade $p$ chegamos em outro de profundidade $p - alturaBalde$).

A linha 5 do código atualiza o valor de $a$ para o próximo vértice do seu caminho até a raiz. Sabemos que $a$ sempre é mais profundo do que $b$ já que as linhas 3 e 4 trocam os vértices $a$ e $b$ caso o segundo seja mais profundo do que o primeiro.

O código termina com uma chamada da função LCA\_Simples já provada no capítulo anterior, que devolve o \LCA\ entre dois vértices de uma árvore.

\subsection{Complexidade}

O algoritmo 4 executa uma simples busca em profundidade com operações de tempo constante (linhas 2-7), e assim sua complexidade de tempo é $O(n+m)$, onde $n$ é aquantidade de vértices e $m$ a quantidade de arestas. A complexidade de espaço é $O(n)$.

Seja $q$ a quantidade de baldes existentes. No laço das linhas 2-5 do algoritmo 5, sendo $q_a$ e $q_b$ as quantidades de baldes acima de $a$ e $b$, respectivamente, em seus caminhos até a raiz, podemos dizer que são executadas $q_a + q_b$ operações. No pior caso, $q_a = q_b = q$, e portanto essa etapa do algoritmo é $O(q)$.

A chamada de LCA\_Simples tem complexidade $O(h)$, onde $h$ é a altura da árvore. Entretanto, ao dividir nossa árvore em baldes, garantimos que qualquer balde terá altura $h = alturaBalde$.

Assim, denotando $p = alturaBalde$, o algoritmo 5 tem complexidade de tempo $O(p + q)$.

\subsection{Tamanho de um balde}

Para que o algoritmo descrito fique de fato mais eficiente do que o do capítulo anterior, devemos escolher bem quantos níveis de profundidade vamos permitir que estejam em um mesmo balde. Ou seja, determinar, para cada balde, o tamanho do intervalo de profundidades que descrevemos anteriormente.

Queremos minimizar a soma $p + q$, (duas fases do algoritmo: avaliar $q$ baldes e percorrer $p$ vértices). Além disso, também vale que $pq = h$, onde $h$ é a altura da árvore. Então, nosso problema resume-se a:

\centerline{minimizar $p + q$}
\centerline{dado que $pq = h$}

\hspace{1cm}

Já que $p \geq 0$ e $q \geq 0$, temos que $(\sqrt{p} - \sqrt{q}))^2 \geq 0$

$\implies p + q \geq 2\sqrt{pq}$
$\implies p + q \geq 2\sqrt{h}$

\hspace{1cm}

Como o objetivo é minimizar $p + q$, queremos que $p + q = 2\sqrt{h}$. Para que isso aconteça, por sua vez, segue que $(\sqrt{p} - \sqrt{q}))^2 = 0 \implies p = q = \sqrt{h}$.

\hspace{1cm}

Vale notar que no pior caso a altura $h$ de uma árvore é igual à quantidade $n$ de vértices que ela possui. Portanto, concluímos que o tamanho ótimo de um balde é sempre $\sqrt{n}$ (com $\sqrt{n}$ baldes), e assim a complexidade de tempo do algoritmo é de fato $O(\sqrt{n})$.

